\documentclass{article}

\usepackage[latin1]{inputenc}
\usepackage[danish]{babel}
\usepackage{float}
\usepackage{fancyhdr}
\usepackage{amsmath}
\usepackage{color}
\usepackage{listings}
\usepackage{graphicx}
\usepackage{pdfpages}
\usepackage{booktabs}
\usepackage{listingsutf8}
%\usepackage{enumitem}
\usepackage[a4paper, top = 1in, bottom = 1in, left=1in,right=1in]{geometry}

\title{Notes for AI}
\author{Peter Heilbo Ratgen}
\date{\today}

\begin{document}
\maketitle
\section{1st February - Introduction}%
\subsection{Basics}
The course is an introduction to the basics of Artificial Intelligence. We will
get an overview of the base of the artificial intelligence methods. We will use
python as a programming language. Labs and support will be done in python.
Prerequisite to the exam is to complete the homework of the lectures.

You should help each other, but coding should be done individually.
\begin{itemize}
  \item Thinking humanly
  \item Acting humanly
  \subitem The Turing test is used to test this. The longer a human can be
  fooled into thing that the human is talking to a human, and not a bot.  This
  could be chatbots acting humanly. You have to test for:
  \begin{itemize}
    \item Natural language processing
    \item Knowledge representation
    \item Automaed reasoning\
    \item Machine learning
  \end{itemize}
    Success depends on deception. Chatbot can use cheap tricks. Mitzuku has
    recently won for the 5th time.
    Computers have a had time with multiple choice questions. Eg the "The large
    ball crashed right through the table because it was made of styrofoam". If
    you replace "styrofoam" with "steel", then the answer is totally different.
    
    \textbf{A better test?} A better Turing test, would be one that can be
    administed and graded by a machine, and more objectivity in the it would not
    depend of subjectity of humans.

  \item Thinking rationally
    \subitem
    It is about the idealized or "right" way of thinking. It is hard to describe
    the world using logical notation. The procedure of applying these local
    statements and deducing them. 
    We also have a though time dealing with uncertainty, representing the gray
    areas.
  \item Acting rationally
    \subitem
    Acting rationally is acting with the goal of achieving the goal, that has
    been set. Utility is about the goal that has been set, whether it is about,
    shortest route, least time or fewest changes in a public transit system.
\end{itemize}
\subsection{Successes of AI}
\begin{itemize}
  \item IBM Watson is an AI created by IBM. IBM is one of the companies that has been
investing in AI for the longest times. 
  \item Self driving cars is one of the successes of AI. This is an example of a
  rational acting. 
  \item Natural language processing is also a great improvement, with speech
technologies and machine translation.
  \item Vision, OCR, handwriting recognition and face detection and recognition.
  \item Mathematics, program solved unsolved conjecture. Also wolfram alpha.
  \item Games
    \subitem Chess(champion beaten in 1997), checkers(solved in 2007), Go
    (beaten for the first time by a Google AI), Google AI beat top StarCraft
    players.
  \item Logistics, scheduling, planning.
    \subitem A lot of the advancement are done by the military. In the 1991 Gulf
    War an AI planned and scheduled for 50,000 vehicles and such.
  \item Robotics
    \subitem Mars rovers, self driving cars, drones, robot soccer, personal
    robotics.
\end{itemize}

\subsection{History}
First model of a neuron was in the 1940's. In the 1950's the turing test was
created, computer chess and machine translation and theorem provers. In 1956 the
'Artificial Intelligence' term was adopted. Herbert Simon said in 1957 that a
computer would beat a machine and a computer would prove a theorem, this
happened 40 years later instead of 10 years.
They realized the problem of machine translation and chatbots was harder to
solve than initially thought. In the late 1960's machine translation was deemed
a failure. Intractability is recognized as fundamental problem. When the
complexity or size of the problem grows exponentially.
There was a boom in expert system in the 1980's and subsequently a bust. Deep
learning, big data and probabilistic learning boomed in the 1990's till now.

The successes now is due to better computers, dominance of statistical
approaches and machine learning, big data and crowd sourcing.

\section{8th of February - Introduction to Python}

Exercises will start at 10:20.
\paragraph{Basics}
Python code is fairly simple and readable. Beginning and ending of blocks is
done purely by indentation. We will use the Python Console for trying out
examples. Variables can change types throughout the program. A variable can
start as a string end as float. We can use \texttt{+} for string concatenation.
We can use triples quotes for strings containing both \texttt{'} and \texttt{"}.

Variables in Python do not have intrinsic types. But assignment does not create
copies, but references. References are deleted by the garbage collector, when
the reference has passed out of scope. Names cannot start with numbers.

We can have multiple assignments. And swapping vars is easy.
\begin{lstlisting}
>>> x, y = 2, 3
>>> y
3
>>> x, y = y, x
>>> y
2
>>> x
3
\end{lstlisting}

\subparagraph{Sequence types}
Sequence types are tuples, strings and lists. In a tuple we can have multiple
types of variables. Tuples are immutable, such that they cannot be changed after
it has been created. Strings are also immutable. Lists are mutable, they can
also have mixed types.
These sequence types have much syntax in common. If we have to change elements
in an immutable tuple or string a new copy has to be created. Lists can be
shrunk or expanded as you go. We assign some different variables:

\begin{lstlisting}[inputencoding=utf8/latin1,basicstyle=\ttfamily,
language=python, keywordstyle=\color{blue}\bfseries, rulecolor=\color{black}]
>>> tu = (23, 'abc', 4.56, (2,3), 'def')
>>> tu
(23, 'abc', 4.56, (2, 3), 'def')
>>> li = ['abc', 34, 4.34, 23]
>>> st = "Hello World"
>>> st
'Hello World'
>>> st = Helllo wordl'
  File "<stdin>", line 1
    st = Helllo wordl'
                ^
SyntaxError: invalid syntax
>>> st = 'Helllo wordl'
>>> st
'Helllo wordl'
>>> st = """THis is a multiple line
... string that uses triple quotes"""
>>> st
'THis is a multiple line\nstring that uses triple quotes'
\end{lstlisting}
We can also have negative indexes, such that -1 is the last character:
\begin{lstlisting}[inputencoding=utf8/latin1,basicstyle=\ttfamily,
language=python, keywordstyle=\color{blue}\bfseries, rulecolor=\color{black}]
>>> st[-1]
's'
>>> st[-2]
'e'
\end{lstlisting}
If want to get the 3 middle elements of the tuple we defined:
\begin{lstlisting}[inputencoding=utf8/latin1,basicstyle=\ttfamily,
language=python, keywordstyle=\color{blue}\bfseries, rulecolor=\color{black}]
>>> tu[1:4]
('abc', 4.56, (2, 3))
\end{lstlisting}
From this we get a \underline{copy} of the selected part of the tuple. If we do
not specify where in the tuple to start, we start from the beginning.
\begin{lstlisting}[inputencoding=utf8/latin1,basicstyle=\ttfamily,
language=python, keywordstyle=\color{blue}\bfseries, rulecolor=\color{black}]
>>> tu[:3]
(23, 'abc', 4.56)
\end{lstlisting}
We can do a copy like this:
\begin{lstlisting}[inputencoding=utf8/latin1,basicstyle=\ttfamily,
language=python, keywordstyle=\color{blue}\bfseries, rulecolor=\color{black}]
>>> tu[:]
(23, 'abc', 4.56, (2, 3), 'def')
\end{lstlisting}

In this example there is a big difference between line 3 and 4.
\begin{lstlisting}[inputencoding=utf8/latin1,basicstyle=\ttfamily,
language=python, keywordstyle=\color{blue}\bfseries, rulecolor=\color{black}]
>>> l3 = ['4', '5']
>>> l4 = ['6', '7']
>>> l3 = l4
>>> l3 = l4[:]
\end{lstlisting}
In line 3 we assign \texttt{l3} we assign to the reference to \texttt{l4}. In
the 4th line we assign \texttt{l3} to a \underline{copy} of \texttt{l4}, such
that changes in \texttt{l4} will not be reflected in \texttt{l3}.

We can use the \texttt{in} operator to check if we have a substring. We can
concat tuples:
\begin{lstlisting}[inputencoding=utf8/latin1,basicstyle=\ttfamily,
language=python, keywordstyle=\color{blue}\bfseries, rulecolor=\color{black}]
>>> tu[:] + tu2[:]
(23, 'abc', 4.56, (2, 3), 'def', 12, 'yeet')
>>> tu + tu2
(23, 'abc', 4.56, (2, 3), 'def', 12, 'yeet')
\end{lstlisting}

\subparagraph{Dictionaries}
Dictionaries can store a mapping between a set of keys and values. We can
create a dictionary with:
\begin{lstlisting}[inputencoding=utf8/latin1,basicstyle=\ttfamily,
language=python, keywordstyle=\color{blue}\bfseries, rulecolor=\color{black}]
>>> d = {'user' : 'bozo', 'pswd' : 1234}
\end{lstlisting}
The values can be anything. We can get the value of a key like this:
\begin{lstlisting}[inputencoding=utf8/latin1,basicstyle=\ttfamily,
language=python, keywordstyle=\color{blue}\bfseries, rulecolor=\color{black}]
>>> d['user']
'bozo'
\end{lstlisting}
We can delete a key:value pair like this:
\begin{lstlisting}[inputencoding=utf8/latin1,basicstyle=\ttfamily,
language=python, keywordstyle=\color{blue}\bfseries, rulecolor=\color{black}]
>>> del d['pswd']
>>> d
{'user': 'bozo'}
\end{lstlisting}








\end{document}

